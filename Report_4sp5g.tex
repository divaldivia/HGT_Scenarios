\documentclass[10pt,a4paper,notitlepage]{report}
\usepackage[utf8]{inputenc}
\usepackage[english]{babel}
\usepackage{amsmath}
\usepackage{amsfonts}
\usepackage{amssymb}
\usepackage{graphicx}
\usepackage[left=2cm,right=2cm,top=2cm,bottom=2cm]{geometry}
\begin{document}
%We will proceed by contradiction. 
%Suppose that **.
%
%Let $V(G)=\{g_{1},g_{2},g_{3},g_{4}\}$. We will construct G with the constrains given by its degree sequence.
%
%Without loss of generality, suppose that:
%	\begin{enumerate}
%		\item	For $g_{3}$ and $g_{4}$,  $d_{in}(g_{3})=d_{in}(g_{4})=3$.
%			
%		Since G is not a multigraph, $\forall i,j \in \{1,..,4\}$, $g_{i}, g_{j}$ have at most one directed edge $e=(g_{i}, g_{j})$ if $i \neq j$, and cannot have an edge $e=(g_{i}, g_{i})$. Hence, the only way of having $d_{in}(g_{i})=3$, in a graph of order 4, is by directing exactly one edge from  each $g_{j}, j \neq i$. This constrain give us with the following subgraph of G :
%		
%		\item For $g_{1}$ and $g_{2}$,  $d_{in}(g_{1})=d_{in}(g_{2})=1$.
%		
%		Observe that the subgraph obtained in 1. generates $d_{out}(g_{1})=d_{out}(g_{2})=2$ which is the out-degree of all the vertices. This means that the edges $(g_{1}, g_{2})$ and $(g_{2}, g_{1})$ cannot exist. So, our only posibilties to construct the in-degree sequence for  $g_{1}$ and $g_{2}$ is directing edges from $g_{3}$ or $g_{4}$.
%		Lets assume that the edges directed to $g_{1}$ and $g_{2}$ comes, without loss of generality, only from $g_{3}$. In this case, $d_{out}(g_{3})=3$ insted of 2. Therefore, we must have one edge from $g_{3}$ and another one from $g_{4}$. With this last constrain, degree sequence is satisficed and give us two possible graphs, $G'$ and $G''$, as candidates for G:
%		
%		
%		Note that $G' \cong G''$ and $G'=G_{1}$. So, in case, $G \cong G_{1}$, which is a contradiction to the initial hypotesis.  
%	\end{enumerate}		
\section*{HGT scenarios: four species and five genes}	

\subsection{Possible scenarios}

There are only two topologies for a four-species tree:
	\begin{center}
		\includegraphics[scale=1]{casos_generales_4sp5g.jpg}
	\end{center}


\subsection{Colored-orthology graphs}

\subsection{Coloring the scenarios}

\subsection{How to identify HGT and correct the orthology graph}
	Let T be a tree with $L_T=\{ g_1, g_2, g_3, g_4, g_5\}$ which correspond to four species $A, B, C$ and $D$, and say that $g_1$ and $g_2$ are in the same species, implying that one of them was acquired via HGT. 
	\\
	How to correct the orthology graph of T:
	\begin{enumerate}
		\item Identify the HGT donor ($h$) and receptor ($h'$):
			\begin{itemize}
				\item Let $G_1$ and $G_2$ be the sets of four genes such that $\{g_1, g_2, h\}$ is a subset.
				\item Generate the colored orthology graphs for $S_1$ and $S_2$ and add them to generate a five-vertex graph with 					two colors.
				\item Identify the bidirectional edges of $g_1$ and $g_2$.
				\item Choose as $h'$, the vertex which has a bi-colored and bidirectional edge.
				\item \textit{Choose as $h$ the other vertex of this bi-colored and bidirectional edge.}
			\end{itemize}
		\item Delete $h'$ and its edges from the five-vertex graph.
		\item Complete the unidirectional edges to bidirectional edges. 
	\end{enumerate}
	

\end{document}